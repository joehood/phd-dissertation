
\chapter{Conclusions and Future Work}\label{chap:future}

\section{Feasibility and Benefits of the QDL Method}

The stated goal of this work is to demonstrate that the proposed QDL method for transient simulation of electrical power systems is both feasible and advantageous over the current state-of-the-art time-slicing numerical integration methods.

As far as feasibility, the results produced by QDL match the benchmark solution reasonably well, and the performance (computational efficiency) is comparable or faster than time-slicing methods. Several simulation systems and many device models have been created and tested using the method, and all were solvable using the method. The test systems include many smaller examples that highlight the particular features of QDL. Ultimately, in order to demonstrate suitability for real-world power system models, the final test in this work is a relatively large (32 state), non-linear power system model with machines, converters, and other common power system devices. From these tests, it is concluded that the QDL method is at least a feasible solution to the transient simulation of large, real-world power systems.

\section{Opportunities for Future Research}

The steady-state behavior of the QDL simulator when simulating large, non-linear systems is problematic. There is significant high frequency noise in the quantized output quantities, particularly affecting the fast electrical quantities (bus voltages and cable currents). There are also lower frequency oscillations in the signal that correspond roughly to the natural mechanical oscillation modes of the system dominated by the machine inertia. It is expected that the noise and oscillations are artifacts from the time delays introduced to the dynamical system from the QDL method. The oscillation and noise do not appear to be damped, and therefore appear as limit cycles in the output. Possible methods of mitigating this behavior are better tuning of the quantization step size, and adding additional inserted latency when the system is detected to be close to steady-state. These mitigation possibilities were investigated in chapter \ref{chap:steadystate}, and proved modestly successful, at least for smaller test systems. Future work should include scaling these proposed solutions, and investigating other possible solutions to the noise and error propagation problem.

In addition to the steady-state issues, problems were also found running transient fault analysis. Typical power system studies include the analysis of faults. Transient fault analysis is important for testing the robustness of the power system and its protection system. For the QDL approach to be a viable alternative to other methods, the ability to properly simulate faults is important. However, performance and technology issues prevented fault scenarios from being included in the paper. QSS integration methods in their current form have a specific disadvantage in simulating faults. Because fault scenarios involve large, abrupt changes to system quantities (a voltage quantity quickly moving from its rated value to near zero, for example), the system states move very quickly through many quantization steps, and these quantization changes cascade widely to the rest of the system causing extremely high computational intensity during the transients of the fault scenarios. Strategies for overcoming or compensating for these issues are required that are beyond the scope of this initial feasibility evaluation. These strategies could include code optimization, memory management, or even modifications to the core QSS algorithms to mitigate the problems. It is not expected that the latest \emph{modified} LIQSS methods (such as mLIQSS1 and mLIQSS2) will have better performance with fault scenarios on power systems models. These updated methods address different problems than those posed by transient fault analysis, and the additional computation steps they require per update are likely to exacerbate this problem. Future work should include upgrading the QDL simulator implementations with the latest LIQSS methods from the most recent literature.
